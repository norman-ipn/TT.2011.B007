\chapter{Conclusiones}

Se presentan las conclusiones a las que se llegaron despu�s del desarrollo del trabajo terminal.

\section{Generales}
Actualmente las interfaces de usuario natural est�n acaparando el mercado de la tecnolog�a desde tel�fonos m�viles hasta computadoras personales intentando hacer de manera m�s sencilla la interacci�n entre ellas y los usuarios, eliminando dispositivos intermediarios y que el usuario pueda manejar programas o sistemas como si estuviera haciendo la actividad regular, como si no contara con el dispositivo, por ejemplo escribir o dibujar.\\\\

Con la realizaci�n de este trabajo terminal se ofrece un sistema con una alternativa al mouse y teclado convencionales y una interacci�n diferente ya que la interfaz no cuenta con botones sino con herramientas f�sicas para la selecci�n de las distintas opciones.\\\\

Trabajando con {\it Kinect}\texttrademark nos dimos cuenta que tan r�pido avanza la tecnolog�a y el desarrollo de la industria computacional ya que cuando se plante� el proyecto no se contaba con documentaci�n fiable ni con los drivers oficiales por parte de {\it Microsoft}\textregistered y en tan solo unos meses ya hab�a bastantes desarrollos con este dispositivo que aumentaron con la liberaci�n del entorno de desarrollo y los drivers oficiales. 


\section{Individuales}
\begin{itemize}
	\item Durante el desarrollo de este trabajo terminal conocimos alternativas al mouse y teclado pero no alguna que ofrezca una interacci�n sin tener contacto con alg�n dispositivo electr�nico. Esta investigaci�n que realizamos para ofrecer una interfaz de usuario natural es un paso m�s para que se realicen sistemas donde ya no se tenga que trabajar con un mouse o teclado, espero que este trabajo funcione como base para que en un futuro pr�ximo las interfaces naturales sean las dominantes en el mercado.\\\\
Adem�s aprendimos la organizaci�n que se debe de tener para el desarrollo de un sistema, nos sirvi� de experiencia para en el futuro sepamos c�mo realizar desde una investigaci�n si es que no contamos con los conocimientos o es una tecnolog�a reciente como lo fue Kinect hasta la implementaci�n de un sistema con las nuevas tecnolog�as.
{\scriptsize S�nchez Ram�rez Gustavo Rafael.}
\end{itemize}



\section{Trabajo a futuro}

Se deja como base un trabajo de investigaci�n para la correcta instalaci�n y configuraci�n de una computadora personal para trabajar con {\it Kinect}\texttrademark y poder realizar una aplicaci�n o un sistema m�s robusto, ya que una de las finalidades de este trabajo era investigaci�n acerca de c�mo manejar {\it Kinect}\texttrademark con una computadora.\\\\
Dirigirlo hacia un sector en espec�fico de la poblaci�n solucionando un problema o proponiendo una alternativa s�lida para desplazar poco a poco a los {\it hardware} directos, teclado y mouse, y no tener que depender  en su totalida de ellos en un sistema.


