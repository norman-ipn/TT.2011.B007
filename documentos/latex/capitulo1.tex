%% Todas las palabras extranjeras con 
%% itálicas {\it texto en itálicas}

\chapter{Introducción}

El presente trabajo muestra el desarrollo de un sistema {\it multi-touch} %% todas las palabras extranjeras
 para crear figuras básicas bidimensionales.

%% Esta parte está pendiente de revisar.... no sé que es lo que decidieron al respecto.
La interacción con el sistema se contará con objetos físicos que representan herramientas para dibujar, además se cuenta con otros objetos como herramientas básicas para la edición del dibujo. 


El desarrollo del sistema está enfocado en la integración de tecnologías, 
en éste caso particular se utilizará la herramienta Kinect™ de Microsoft® conectada a una computadora personal, 
además el sistema contará con reconocimiento de patrones para la selección de herramientas de dibujo o de edición del mismo.

%%
%%
\section{Kinect}

Kinect™ es un dispositivo que combina una cámara RGB, un sensor de profundidad y un arreglo de micrófonos[1].

Como podemos observar en la Tabla 1, se listan los elementos principales de Kinect junto con su función.

\section{OpenNI}

\section{OpenCV}

\section{Reconocimiento de Patrones}

\section{Etiquetas} %% Anteriormente TAG's

\subsection{reacTIVision}

\subsection{Código QR}
\subsection{ARToolkit}

\section{Metodología}

El desarrollo del proyecto se realizó aplicando el modelo incremental. %% Falta refenrencia al modelo incremental  
Esta metodología tiene la ventaja de ser dinámica y flexible, además permite usar las salidas de las etapas precedentes, 
como entradas en las etapas sucesivas y facilita corregir cualquier error detectado o llevar a cabo mejoras en 
los distintos productos que se generan a lo largo de su aplicación[13]. 

El uso de esta metodología dentro del desarrollo del proyecto proporcionó:
\begin{itemize}
\item Definición de actividades a llevarse a cabo en el tiempo de realización del Trabajo Terminal.
\item Unificación de criterios en la organización para el desarrollo del proyecto.
\item Puntos de control y revisión.
\item Seguimiento de secuencias ascendentes o descendentes en las etapas del desarrollo.
\item Cumplimiento de etapas o fases en paralelo, por lo que es más flexible que la estructurada.
\end{itemize}

\subsection{Paradigma}

\section{Objetivos}

\subsection{General}
\subsection{Particulares}

\section{Justificación}

\subsection{Estado del arte}



