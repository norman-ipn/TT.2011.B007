\chapter{Anexo. Manual de Instalación OpenNI/OpenCV}

\section{Introducción}

El objetivo de este manual es explicar la instalación del {\it driver} OpenNi y del framework OpenCV, 
para la utilización del dispositivo kinect, en cualquier equipo de cómputo, 
en este documento se citan los requerimientos previos con los que el usuario debe contar. 
También se explican los pasos necesarios para una correcta instalación.
El {\it driver} y framework son multiplataforma, con soporte para Linux, Windows ó Mac OS.
Este manual sólo describe el proceso de instalación y configuración para una distribución de Linux.
El manual incluye las versiones de los paquetes que se usaron, otras versiones pueden no funcionar
 pero puede servir como previo para la búsqueda de información.
 
 
\section{Requisitos previos}

Tener instalada una distribución de Linux. 
Este manual fue probado con Ubuntu 11.04 en 32 bits.

\subsection{Sistema Operativo}

En caso de no tener una distrubución Linux los pasos básicos se pueden resumir en:
\begin{enumerate}
\item Se inserta el disco y se procede a instalar
\item Después nos dará una serie de opciones, de la cual usaremos la opción de algo mas
\item Se escoge la partición mas grande  con la que se cuenta y se reduce el espacio
\item Nos quedara un espacio libre, el cual fue reducido en la partición, este espacio se usara para
crear una nueva partición.
\item Se crea la partición, que sea de tipo lógica
\item Por último, se le da instalar y seguimos las instrucciones del asistente
\end{enumerate}

\subsection{Gestor de paquetes}

Las distribuciones incluyen gestores de paquetes que ayudan en la instalación de bibliotecas de funciones
pero en caso de no tener un gestor instalador se pueden seguir los siguientes pasos:

\begin{enumerate}
\item Se abre una terminal.
\item Se escribe el siguiente comando:
\begin{verbatim}
sudo apt-get install synaptic
\end{verbatim}
\end{enumerate}


 
