\chapter{Anexo. Manual de Instalaci�n OpenNI/OpenCV}

\section{Introducci�n}

El objetivo de este manual es explicar la instalaci�n del {\it driver} OpenNi y del framework OpenCV, 
para la utilizaci�n del dispositivo kinect, en cualquier equipo de c�mputo, 
en este documento se citan los requerimientos previos con los que el usuario debe contar. 
Tambi�n se explican los pasos necesarios para una correcta instalaci�n.
El {\it driver} y framework son multiplataforma, con soporte para Linux, Windows � Mac OS.
Este manual s�lo describe el proceso de instalaci�n y configuraci�n para una distribuci�n de Linux.
El manual incluye las versiones de los paquetes que se usaron, otras versiones pueden no funcionar
 pero puede servir como previo para la b�squeda de informaci�n.
 
 
\section{Requisitos previos}

Tener instalada una distribuci�n de Linux. 
Este manual fue probado con Ubuntu 11.04 en 32 bits.

\subsection{Sistema Operativo}

En caso de no tener una distrubuci�n Linux los pasos b�sicos se pueden resumir en:
\begin{enumerate}
\item Se inserta el disco y se procede a instalar
\item Despu�s nos dar� una serie de opciones, de la cual usaremos la opci�n de algo mas
\item Se escoge la partici�n mas grande  con la que se cuenta y se reduce el espacio
\item Nos quedara un espacio libre, el cual fue reducido en la partici�n, este espacio se usara para crear una nueva partici�n.
\item Se crea la partici�n, que sea de tipo l�gica
\item Por �ltimo, se le da instalar y seguimos las instrucciones del asistente
\end{enumerate}

\subsection{Gestor de paquetes}

Las distribuciones incluyen gestores de paquetes que ayudan en la instalaci�n de bibliotecas de funciones
pero en caso de no tener un gestor instalador se pueden seguir los siguientes pasos:

\begin{enumerate}
\item Se abre una terminal.
\item Se escribe el siguiente comando:
\begin{verbatim}
sudo apt-get install synaptic
\end{verbatim}
\end{enumerate}


 
