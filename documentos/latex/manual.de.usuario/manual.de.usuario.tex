\documentclass{book}
\usepackage[spanish]{babel}
\usepackage[utf8]{inputenc}

\begin{document}

\chapter{Introducción}

El objetivo de este manual es explicarle al usuario, como puede interactuar con el 
Sistema de Dibujo en 2D  Multi-touch con Reconocimiento de Patrones.
En este documento se citan los requerimientos previos con los que el usuario debe contar. 
Así mismo se explica como el usuario podrá usar cada una de las partes del sistema y darle un buen funcionamiento al sistema.
El sistema permite al usuario interactuar con su dedo e imágenes que representan las tareas permitidas en el sistema, 
se cuenta con una base, la cual sirve de soporte al dispositivo sensor (Kinect(TM)).
Este sensor se coloca en la parte inferior de la base, mientras que en la superficie está un vidrio, 
donde se colocarán las imágenes denominadas {\it tags} o ya sea el dedo que será el que realice el trazo.

\section{Funciones del sistema}

El sistema ofrece dos tipos de actividades, las de dibujo y las de modificación, 
las cuales se enlistan a continuación:

\begin{itemize}
\item Para dibujar:

 \begin{itemize}
  \item Mano Alzada
  \item Línea
  \item Triángulo
  \item Cuadrado
  \item Pentágono
  \item Hexágono
  \item Elipse
 \end{itemize}

\item Para modificar una figura:
  \begin{itemize}
    \item Mover
    \item Cambiar  de color (Verde, Rojo, Azul y Negro).
    \item Cambiar de tamaño (50\%, 75\%, 125\% y 150\%).
  \end{itemize}

\end{itemize}


\end{document}
