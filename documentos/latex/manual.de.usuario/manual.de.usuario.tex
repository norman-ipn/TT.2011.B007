\documentclass{book}
\usepackage[spanish]{babel}
\usepackage[utf8]{inputenc}

\begin{document}

\chapter{Introducción}

El objetivo de este manual es explicarle al usuario, como puede interactuar con el Sistema de Dibujo en 2D  Multi-touch con Reconocimiento de Patrones, en este documento se citan los requerimientos previos  con los que el usuario debe contar. Así mismo se explica, como el usuario podrá usar cada una de las partes del sistema, y darle un buen funcionamiento al sistema.
El sistema permite al usuario interactuar con su dedo e imágenes que representan las tareas permitidas en el sistema, se cuenta con una base, la cual sirve de soporte al kinect, eso es respecto en la parte inferior de la base, mientras que en la superficie esta un vidrio, donde se colocaran las imágenes denominadas tags o ya sea el dedo que será el que realice el trazo.

Las funciones con que cuenta el sistema son las siguientes:

\begin{itemize}
\item Para dibujar se tienen:

 \begin{itemize}
  \item Mano Alzada
  \item Línea
  \item Triángulo
  \item Cuadrado
  \item Pentágono
  \item Hexágono
  \item Elipse
 \end{itemize}

\item Para modificar una figura se tienen:
  \begin{itemize}
    \item Mover
    \item Cambiar  de color (Verde, Rojo, Azul y Negro).
    \item Cambiar de tamaño (50\%, 75\%, 125\% y 150\%).
  \end{itemize}

\end{itemize}


\end{document}
